% !TeX root = thesis.tex
%--------------------------------------------------------------------%
%
% Template TA LaTeX Teknik Informatika ITERA.
% Editor: Radhinka Bagaskara, Martin C.T. Manullang, iwawiwi
% Version 2022-0.1
%
% Berdasarkan "Templat LaTeX Tesis Informatika ITB" oleh Petra Barus & Peb Ruswono Aryan
% https://github.com/petrabarus/if-itb-latex
%--------------------------------------------------------------------%
%
% Berkas ini berisi struktur utama dokumen LaTeX yang akan dibuat.
%
%--------------------------------------------------------------------%

\documentclass[12pt, a4paper, onecolumn, oneside, final]{report}

\input{config/if-itb-thesis.sty}

\makeatletter

\makeatother

\bibliography{references}

\begin{document}
    \sloppy % mencegah text overflow.
    %Basic configuration
\title{Judul Tugas Akhir} 	% Judul Tugas Akhir
\titleEN{Thesis Title}                  % Thesis Title
\author{Nama Mahasiswa}		% Nama Mahasiswa
\nim{1234567890}				% NIM Mahasiswa
\dosbingA%
    {Nama dan Gelar Pembimbing I}%	% Nama Dosen Pembimbing 1
    {NIP. 123456789}				% NIP Dosen Pembimbing 1
\dosbingB%
    {Nama dan Gelar Pembimbing II}%	% Nama Dosen Pembimbing 2
    {NIP. 123456789}				% NIP Dosen Pembimbing 2

\pagenumbering{roman}
\setcounter{page}{0}

\input{chapters/ch00/cover-hard} % Hardcover
\input{chapters/ch00/cover-soft} % Softcover
\input{chapters/ch00/approval}
\input{chapters/ch00/statement}
\input{chapters/ch00/publication}

\input{chapters/ch00/abstract-id}
\clearpage

\begin{minipage}{\textwidth}
	\singlespacing{
	\textbf{\thetitleEN}\\
	\mbox{\theauthor \ (\printnim)}\\
	Pembimbing I \printnamadosbinga\\
	Pembimbing II \printnamadosbingb\\
}
\end{minipage}

%\chapter*{ABSTRAK}
\normalsize \bfseries \centering \MakeUppercase{Abstract}
\phantomsection% 
\addcontentsline{toc}{chapter}{Abstract}
\\[2\baselineskip]

%taruh abstrak bahasa inggris di sini
\justifying \normalfont \normalsize{

}

\textbf{Keyword}: Keyword 1, Keyword 2
\clearpage
\input{chapters/ch00/motto}
\input{chapters/ch00/dedicated}
\input{chapters/ch00/forewords}

\tableofcontents
\listoffigures
\listoftables
\input{chapters/ch00/symbols}

    %----------------------------------------------------------------%
    % Konfigurasi Bab
    %----------------------------------------------------------------%
    \renewcommand{\chaptername}{BAB}
    % Bab: Arabic
    \renewcommand{\thechapter}{\Roman{chapter}}
    % Sub-bab: Roman
    \renewcommand\thesection{\arabic{chapter}.\arabic{section}}
    
    % Setting supaya nomor halaman pertama dengan "chapter"
    % berada di tengah bawah
    \fancypagestyle{plain}{%
    	\fancyhf{}%
    	\renewcommand{\headrulewidth}{0pt}
    	\fancyhead[]{}
    	\fancyfoot[C]{\thepage}
    }
    %----------------------------------------------------------------%

    %----------------------------------------------------------------%
    % Daftar Bab
    % Untuk menambahkan daftar bab, buat berkas bab misalnya `chapter-6` di direktori `chapters`, dan masukkan ke sini.
    %----------------------------------------------------------------%

    % Reset penomoran halaman menjadi 1
    \clearpage
    \setcounter{page}{1}
    \pagenumbering{arabic}

    \justifying
    %Basic configuration
\title{Judul Tugas Akhir} 	% Judul Tugas Akhir
\titleEN{Thesis Title}                  % Thesis Title
\author{Nama Mahasiswa}		% Nama Mahasiswa
\nim{1234567890}				% NIM Mahasiswa
\dosbingA%
    {Nama dan Gelar Pembimbing I}%	% Nama Dosen Pembimbing 1
    {NIP. 123456789}				% NIP Dosen Pembimbing 1
\dosbingB%
    {Nama dan Gelar Pembimbing II}%	% Nama Dosen Pembimbing 2
    {NIP. 123456789}				% NIP Dosen Pembimbing 2

\pagenumbering{roman}
\setcounter{page}{0}

\input{chapters/ch00/cover-hard} % Hardcover
\input{chapters/ch00/cover-soft} % Softcover
\input{chapters/ch00/approval}
\input{chapters/ch00/statement}
\input{chapters/ch00/publication}

\input{chapters/ch00/abstract-id}
\clearpage

\begin{minipage}{\textwidth}
	\singlespacing{
	\textbf{\thetitleEN}\\
	\mbox{\theauthor \ (\printnim)}\\
	Pembimbing I \printnamadosbinga\\
	Pembimbing II \printnamadosbingb\\
}
\end{minipage}

%\chapter*{ABSTRAK}
\normalsize \bfseries \centering \MakeUppercase{Abstract}
\phantomsection% 
\addcontentsline{toc}{chapter}{Abstract}
\\[2\baselineskip]

%taruh abstrak bahasa inggris di sini
\justifying \normalfont \normalsize{

}

\textbf{Keyword}: Keyword 1, Keyword 2
\clearpage
\input{chapters/ch00/motto}
\input{chapters/ch00/dedicated}
\input{chapters/ch00/forewords}

\tableofcontents
\listoffigures
\listoftables
\input{chapters/ch00/symbols}
    %Basic configuration
\title{Judul Tugas Akhir} 	% Judul Tugas Akhir
\titleEN{Thesis Title}                  % Thesis Title
\author{Nama Mahasiswa}		% Nama Mahasiswa
\nim{1234567890}				% NIM Mahasiswa
\dosbingA%
    {Nama dan Gelar Pembimbing I}%	% Nama Dosen Pembimbing 1
    {NIP. 123456789}				% NIP Dosen Pembimbing 1
\dosbingB%
    {Nama dan Gelar Pembimbing II}%	% Nama Dosen Pembimbing 2
    {NIP. 123456789}				% NIP Dosen Pembimbing 2

\pagenumbering{roman}
\setcounter{page}{0}

\input{chapters/ch00/cover-hard} % Hardcover
\input{chapters/ch00/cover-soft} % Softcover
\input{chapters/ch00/approval}
\input{chapters/ch00/statement}
\input{chapters/ch00/publication}

\input{chapters/ch00/abstract-id}
\clearpage

\begin{minipage}{\textwidth}
	\singlespacing{
	\textbf{\thetitleEN}\\
	\mbox{\theauthor \ (\printnim)}\\
	Pembimbing I \printnamadosbinga\\
	Pembimbing II \printnamadosbingb\\
}
\end{minipage}

%\chapter*{ABSTRAK}
\normalsize \bfseries \centering \MakeUppercase{Abstract}
\phantomsection% 
\addcontentsline{toc}{chapter}{Abstract}
\\[2\baselineskip]

%taruh abstrak bahasa inggris di sini
\justifying \normalfont \normalsize{

}

\textbf{Keyword}: Keyword 1, Keyword 2
\clearpage
\input{chapters/ch00/motto}
\input{chapters/ch00/dedicated}
\input{chapters/ch00/forewords}

\tableofcontents
\listoffigures
\listoftables
\input{chapters/ch00/symbols}
    %Basic configuration
\title{Judul Tugas Akhir} 	% Judul Tugas Akhir
\titleEN{Thesis Title}                  % Thesis Title
\author{Nama Mahasiswa}		% Nama Mahasiswa
\nim{1234567890}				% NIM Mahasiswa
\dosbingA%
    {Nama dan Gelar Pembimbing I}%	% Nama Dosen Pembimbing 1
    {NIP. 123456789}				% NIP Dosen Pembimbing 1
\dosbingB%
    {Nama dan Gelar Pembimbing II}%	% Nama Dosen Pembimbing 2
    {NIP. 123456789}				% NIP Dosen Pembimbing 2

\pagenumbering{roman}
\setcounter{page}{0}

\input{chapters/ch00/cover-hard} % Hardcover
\input{chapters/ch00/cover-soft} % Softcover
\input{chapters/ch00/approval}
\input{chapters/ch00/statement}
\input{chapters/ch00/publication}

\input{chapters/ch00/abstract-id}
\clearpage

\begin{minipage}{\textwidth}
	\singlespacing{
	\textbf{\thetitleEN}\\
	\mbox{\theauthor \ (\printnim)}\\
	Pembimbing I \printnamadosbinga\\
	Pembimbing II \printnamadosbingb\\
}
\end{minipage}

%\chapter*{ABSTRAK}
\normalsize \bfseries \centering \MakeUppercase{Abstract}
\phantomsection% 
\addcontentsline{toc}{chapter}{Abstract}
\\[2\baselineskip]

%taruh abstrak bahasa inggris di sini
\justifying \normalfont \normalsize{

}

\textbf{Keyword}: Keyword 1, Keyword 2
\clearpage
\input{chapters/ch00/motto}
\input{chapters/ch00/dedicated}
\input{chapters/ch00/forewords}

\tableofcontents
\listoffigures
\listoftables
\input{chapters/ch00/symbols}
    %Basic configuration
\title{Judul Tugas Akhir} 	% Judul Tugas Akhir
\titleEN{Thesis Title}                  % Thesis Title
\author{Nama Mahasiswa}		% Nama Mahasiswa
\nim{1234567890}				% NIM Mahasiswa
\dosbingA%
    {Nama dan Gelar Pembimbing I}%	% Nama Dosen Pembimbing 1
    {NIP. 123456789}				% NIP Dosen Pembimbing 1
\dosbingB%
    {Nama dan Gelar Pembimbing II}%	% Nama Dosen Pembimbing 2
    {NIP. 123456789}				% NIP Dosen Pembimbing 2

\pagenumbering{roman}
\setcounter{page}{0}

\input{chapters/ch00/cover-hard} % Hardcover
\input{chapters/ch00/cover-soft} % Softcover
\input{chapters/ch00/approval}
\input{chapters/ch00/statement}
\input{chapters/ch00/publication}

\input{chapters/ch00/abstract-id}
\clearpage

\begin{minipage}{\textwidth}
	\singlespacing{
	\textbf{\thetitleEN}\\
	\mbox{\theauthor \ (\printnim)}\\
	Pembimbing I \printnamadosbinga\\
	Pembimbing II \printnamadosbingb\\
}
\end{minipage}

%\chapter*{ABSTRAK}
\normalsize \bfseries \centering \MakeUppercase{Abstract}
\phantomsection% 
\addcontentsline{toc}{chapter}{Abstract}
\\[2\baselineskip]

%taruh abstrak bahasa inggris di sini
\justifying \normalfont \normalsize{

}

\textbf{Keyword}: Keyword 1, Keyword 2
\clearpage
\input{chapters/ch00/motto}
\input{chapters/ch00/dedicated}
\input{chapters/ch00/forewords}

\tableofcontents
\listoffigures
\listoftables
\input{chapters/ch00/symbols}
    %Basic configuration
\title{Judul Tugas Akhir} 	% Judul Tugas Akhir
\titleEN{Thesis Title}                  % Thesis Title
\author{Nama Mahasiswa}		% Nama Mahasiswa
\nim{1234567890}				% NIM Mahasiswa
\dosbingA%
    {Nama dan Gelar Pembimbing I}%	% Nama Dosen Pembimbing 1
    {NIP. 123456789}				% NIP Dosen Pembimbing 1
\dosbingB%
    {Nama dan Gelar Pembimbing II}%	% Nama Dosen Pembimbing 2
    {NIP. 123456789}				% NIP Dosen Pembimbing 2

\pagenumbering{roman}
\setcounter{page}{0}

\input{chapters/ch00/cover-hard} % Hardcover
\input{chapters/ch00/cover-soft} % Softcover
\input{chapters/ch00/approval}
\input{chapters/ch00/statement}
\input{chapters/ch00/publication}

\input{chapters/ch00/abstract-id}
\clearpage

\begin{minipage}{\textwidth}
	\singlespacing{
	\textbf{\thetitleEN}\\
	\mbox{\theauthor \ (\printnim)}\\
	Pembimbing I \printnamadosbinga\\
	Pembimbing II \printnamadosbingb\\
}
\end{minipage}

%\chapter*{ABSTRAK}
\normalsize \bfseries \centering \MakeUppercase{Abstract}
\phantomsection% 
\addcontentsline{toc}{chapter}{Abstract}
\\[2\baselineskip]

%taruh abstrak bahasa inggris di sini
\justifying \normalfont \normalsize{

}

\textbf{Keyword}: Keyword 1, Keyword 2
\clearpage
\input{chapters/ch00/motto}
\input{chapters/ch00/dedicated}
\input{chapters/ch00/forewords}

\tableofcontents
\listoffigures
\listoftables
\input{chapters/ch00/symbols}
    %----------------------------------------------------------------%

    % Daftar pustaka
    \renewcommand{\bibname}{Daftar Pustaka}
    \phantomsection% 
    \addcontentsline{toc}{chapter}{Daftar Pustaka}
    \printbibliography
    

    % Index
    \appendix

    \addcontentsline{toc}{part}{Lampiran}
    \part*{Lampiran}

    \input{chapters/ch99/appendix-1.tex}
    \input{chapters/ch99/appendix-2.tex}

\end{document}
